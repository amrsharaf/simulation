%%%%%%%%%%%%%%%%%%%%%%%%%%%%%%%%%%%%%%%%%%%%%%
% Lab report template.
%%%%%%%%%%%%%%%%%%%%%%%%%%%%%%%%%%%%%%%%%%%%%%

\documentclass[aps,letterpaper,10pt]{revtex4}

\usepackage{subfigure}
\usepackage{wrapfig}
\usepackage{graphicx} % For images
\usepackage{float}    % For tables and other floats
\usepackage{verbatim} % For comments and other
\usepackage{amsmath}  % For math
\usepackage{amssymb}  % For more math
\usepackage{fullpage} % Set margins and place page numbers at bottom center
\usepackage{listings} % For source code
\usepackage{subfig}   % For subfigures
\usepackage[usenames,dvipsnames]{color} % For colors and names
%\usepackage[pdftex]{hyperref}           % For hyperlinks and indexing the PDF
%\hypersetup{ % play with the different link colors here
%    colorlinks,
%    citecolor=blue,
%    filecolor=blue,
%    linkcolor=blue,
%    urlcolor=blue % set to black to prevent printing blue links
%}

\definecolor{mygrey}{gray}{.96} % Light Grey
\lstset{ 
    language= Matlab,              % choose the language of the code ("language=Verilog" is %popular as well)
   tabsize=3,    						  % sets the size of the tabs in spaces (1 Tab %is replaced with 3 spaces)
	basicstyle=\tiny,               % the size of the fonts that are used for the code
	numbers=left,                   % where to put the line-numbers
	numberstyle=\tiny,              % the size of the fonts that are used for the line-numbers
	stepnumber=2,                   % the step between two line-numbers. If it's 1 each line will %be numbered
	numbersep=5pt,                  % how far the line-numbers are from the code
	backgroundcolor=\color{mygrey}, % choose the background color. You must add %\usepackage{color}
	%showspaces=false,              % show spaces adding particular underscores
	%showstringspaces=false,        % underline spaces within strings
	%showtabs=false,                % show tabs within strings adding particular underscores
	frame=single,	                 % adds a frame around the code
	tabsize=3,	                    % sets default tabsize to 2 spaces
	captionpos=b,                   % sets the caption-position to bottom
	breaklines=true,                % sets automatic line breaking
	breakatwhitespace=false,        % sets if automatic breaks should only happen at whitespace
	%escapeinside={\%*}{*)},        % if you want to add a comment within your code
	commentstyle=\color{BrickRed}   % sets the comment style
}



% TITLE PAGE CONTENT %%%%%%%%%%%%%%%%%%%%%%%%
% Remember to fill this section out for each
% lab write-up.
%%%%%%%%%%%%%%%%%%%%%%%%%%%%%%%%%%%%%%%%%%%%%
\newcommand{\labno}{01}
\newcommand{\labtitle}{Introduction to Network Performance Analysis}
\newcommand{\authorname}{Ahmed Mohsen (14), Amr Sharaf (42)}
\newcommand{\professor}{Dr.Sahar Ghanem, Eng.Ahmed Essam}
\newcommand{\classno}{CS433: Performance Evaluation}
% END TITLE PAGE CONTENT %%%%%%%%%%%%%%%%%%%%

% Make units a little nicer looking and faster to type
\newcommand{\Hz}{\textsl{Hz}}
\newcommand{\KHz}{\textsl{KHz}}
\newcommand{\MHz}{\textsl{MHz}}
\newcommand{\GHz}{\textsl{GHz}}
\newcommand{\ns}{\textsl{ns}}
\newcommand{\ms}{\textsl{ms}}
\newcommand{\s}{\textsl{s}}

\begin{document}  % START THE DOCUMENT!


% TITLE PAGE %%%%%%%%%%%%%%%%%%%%%%%%%%%%%%%%%%%%%%
% If you'd like to change the content of this,
% do it in the "TITLE PAGE CONTENT" directly above
% this message
%%%%%%%%%%%%%%%%%%%%%%%%%%%%%%%%%%%%%%%%%%%%%%%%%%%
\begin{titlepage}
\begin{center}
\includegraphics[width=2cm]{Logo_Alexandria_University.jpg}\\
{\LARGE \textsc{Assignment No. \labno:} \\ \vspace{4pt}}
{\Large \textsc{\labtitle} \\ \vspace{4pt}} 
\rule[13pt]{\textwidth}{1pt} \\ \vspace{150pt}
{\large By: \authorname \\ \vspace{10pt}
Instructors: \professor \\ \vspace{10pt}
\classno \\ \vspace{10pt}
\today}
\end{center}
\end{titlepage}
% END TITLE PAGE %%%%%%%%%%%%%%%%%%%%%%%%%%%%%%%%%%
%\tableofcontents
%\newpage


%%%%%%%%%%%%%%%%%%%%%%%%%%%%%%
%%%%%%%%%%%%%%%%%%%%%%%%%%%%%%
\section{Introduction}
%No Text Here
%%%%%%%%%%%%%%%%%%%%%%%%%%%%%%%
\subsection{Purpose}
The goal of this assignment is to apply some statistical methods to analyze experimental data related to network performance. It is required to use the Iperf tool to analyze and compare different types of network connections. These types include the following: \begin{itemize}
\item LAN connection between two hosts.
\item WLAN connection between two hosts.
\item Remote Internet connection between two hosts not in the same network. (ADSL
connection or a Dial up connection. Testing both is a plus.)
\end{itemize}
\vspace{3mm} % I use this to seperate the paragraphs a bit.

The comparison should be based on the following:
\begin{itemize}
\item TCP connection performance analysis: It is required to obtain the bandwidth
of a TCP connection established on each type of the networks listed above.
\item UDP connection performance analysis: It is required to study the behavior of
the connection throughput and loss ratio using a UDP connection under different
data rates. Through this study, the maximum achievable throughput should be reported. This estimate should be compared to the connection bandwidth obtained by the TCP connection.
\end{itemize}
\vspace{3mm}  

%%%%%%%%%%%%%%%%%%%%%%%%%%%%%%
\subsection{Measurement Tool}
In this assignment, The Iperf tool has been used to collect the required data. Iperf is a tool for measuring TCP and UDP performance over a network.
%%%%%%%%%%%%%%%%%%%%%%%%%%%%%%
\subsection{Procedure}
    \subsubsection{TCP Connection}
        \paragraph{LAN Network}    
        	\begin{enumerate}
        		\item Connect two machines using a network cable.
        		\item Choose one machine as a \textbf{server} and the other as a \textbf{client}.
        		\item At the server machine run \textit{ipconfig command} to obtain server IP address.
        		\item Run Iperf on server machine with a predefined port number \textit{e.g. 5001}. 
        		\item Run Iperf on client machine with the predefined port number and the server IP address.
        		\item repeat the above step n times where n bigger than 30
        	\end{enumerate}
\vspace{3mm}  
        \paragraph{WLAN Connection}    
            \begin{enumerate}
        		\item Setup wireless ad hoc network between two machines.
        		\item Choose one machine as a \textbf{server} and the other as a \textbf{client}.
        		\item At the server machine run \textit{ipconfig command} to obtain server IP address.
        		\item Run Iperf on server machine with a predefined port number \textit{e.g. 5001}. 
        		\item Run Iperf on client machine with the predefined port number and the server IP address.
        		\item repeat the above step n times where n bigger than 30
        	\end{enumerate}
\vspace{3mm}  
        \paragraph{ADSL Connection}   
            \begin{enumerate}
            	\item Configure port forwarding on the server machine \url{http://www.portforward.com/}.
                \item Get the external IP address of the server machine using: \url{http://www.whatismyip.com/}
        		\item Run Iperf on server machine with a predefined port number \textit{i.e. the port configured in step 1}. 
        		\item Run Iperf on client machine with the predefined port number and the server IP address.
        		\item repeat the above step n times where n bigger than 30
        	\end{enumerate}
\vspace{3mm}  
        \paragraph{Dail up Connection}   
            \begin{enumerate}
                \item Connect server machine to a dail up network 
                \item Get the external IP address of the server machine using: \url{http://www.whatismyip.com/}.
        		\item Run Iperf on server machine with a predefined port number . 
        		\item repeat step 1 for client machine.
        		\item repeat the above step n times where n bigger than 30
        	\end{enumerate}

    \subsubsection{UDP Connection}
	Same procedures as TCP. However, the data rate has been changed to measure different throughput values. For each data rate, the experiment has been run for three times and the average results was recorded. This is done to avoid any noise in the resulting data. The packets loss ratio has also been recorded for UDP connections.
%%%%%%%%%%%%%%%%%%%%%%%%%%%%%%
%%%%%%%%%%%%%%%%%%%%%%%%%%%%%%
%%%%%%%%%%%%%%%%%%%%%%%%%%%%%%
\newpage
\section{Performance Analysis Procedure}
    \subsection{TCP Connection}
This section will describe the analytical steps done on the experiment data of the TCP connection\\
Definitions:
    \begin{enumerate}
        \item x = sample collected data
        \item n = sample size of collected data
        \item $\bar{x}$ = sample mean 
        \item s = sample deviation 
        \item r = accuracy of number of samples required to follow normal distribution 
        \item CI = confedence IntervaL 
    \end{enumerate}
    \subsubsection{The mean bandwidth over several trials}
            \[
            \bar{x} = \sum_{i=1}^n{\frac{x_{i}}{n}}
            \]
        \subsubsection{The 95\% confidence intervals for the mean bandwidth}
            \[
            n > 30
            \]
            \[
            \bar{x} \sim N(\mu , \frac{s}{\sqrt{n}})
            \]
            \[
            z_{1-\frac{\alpha}{2}} = z_{0.975} = 1.96 \cdots Normal Table
            \]
            \[
            s = \frac{1}{n-1}\sum_{i=1}^n{(x_{i} - \bar{x})^2}
            \]
            \[
            CI = \bar{x} \mp z_{1-\frac{\alpha}{2}}\frac{s}{\sqrt{n}}
            \]
        \subsubsection{the number of samples required to obtain the mean bandwidth with 1\% accuracy}
            \[
            r = 1
            \]
            \[
            z = z_{1-\frac{\alpha}{2}} = 1.96
            \]
            \[
            N_{sample} = \lceil (\frac{100*z*s}{r*\bar{x}})^2 \rceil
            \]
           

\newpage
\section{Experiment Data}
This section will consist of the actual analysis of the collected data.  \vspace{5mm}
    \subsection{TCP Connection}
    This matlab script is used to generate the required data in the analysis of TCP connection    
    	\lstinputlisting{tcp.m}
    	\vspace{3mm}
        \newpage
        \subsubsection{LAN Network}
            \paragraph{Readings}
                \begin{center}
                    \begin{tabular}{ ||l || c | c | c | c | c | c | c | c | c | c | c | c | c | c | c | }
                    \hline
                    Trial number & 1 & 2 & 3 & 4 & 5 & 6 & 7 & 8 & 9 & 10 & 11 & 12 & 13 & 14 & 15 \\ \hline
                    Bandwidth(Mbps) & 81.8 & 82.2 & 78.1 & 74.7 & 75.7 & 75.6 & 77.5 & 75.6 & 74.1 & 75.4 & 76.7 & 76.7 & 74.8 & 74.8 & 77.6\\
                    \hline   
                    \hline
                    Trial number & 16 & 17 & 18 & 19 & 20 & 21 & 22 & 23 & 24 & 25 & 26 & 27 & 28 & 29 & 30 \\ \hline 
                    Bandwidth(Mbps) & 77.2 & 75.0 & 76.0 & 75.8 & 74.0 & 76.6 & 77.2 & 75.5 & 82.0 & 76.9 & 75.2 & 76.5 & 76.5 & 76.3 & 76.0\\     
                    \hline    
                    \hline
                    Trial number & 31 & 32 & 33 & 34 & 35 & 36 & 37 & 38 & 39 & 40 & & & & & \\ \hline
                    Bandwidth(Mbps) & 76.6 & 75.3 & 75.3 & 76.5 & 75.9 & 76.3 & 74.1 & 75.4 & 74.7 & 73.5 & & & & &\\
                    \hline
                    \end{tabular}
                \end{center}
                \vspace{3mm}
            \paragraph{Results}
                \begin{itemize}
                    \item $\bar{x} = 76.3 Mbps$
                    \item $s = 1.94$
                    \item $Confidence$ $interval = (75.7 , 76.9)$
                    \item $N_{normal} = 26$
                \end{itemize}
            \vspace{3mm}           
                         \begin{figure}[htp]
                \begin{center}
                    \subfigure[Qunatiltle quantile plot]{
                    \includegraphics[scale=0.5]{tcp-lan.png}
                    }
                    \subfigure[Histogram plot]{
                    \includegraphics[scale=0.5]{tcp-lan_hist.png}
                    }                        
                \end{center}    
            \end{figure}  
            \paragraph{Observations}
                \begin{itemize}
                        \item Bandwidth is faster than usual DSL 
                        \item according quantile quantile plot the sample data follow noraml distribution
                        \item packet Delay is minimal \textit{i.e.} ad hoc network is used which eliminates the need of a router and therefore the queuing delay and processing delay \\
                        \[
                        d_{nodal} = d_{processing} + d_{queue} + d_{transmission} + d_{propagation}
                        \] 
                        \[
                        d_{processing} = 0 , d_{queue} = 0
                        \] 
                        \item When packet size increase results become more accurate
                        \item the material of the connecting wire affects the limit of the bandwidth
                    \end{itemize}
        \newpage            
        \subsubsection{WLAN Network}
            \paragraph{Readings}
                \begin{center}
                    \begin{tabular}{ ||l || c | c | c | c | c | c | c | c | c | c | c | c | c | c | c | }
                    \hline
                    Trial number & 1 & 2 & 3 & 4 & 5 & 6 & 7 & 8 & 9 & 10 & 11 & 12 & 13 & 14 & 15 \\ \hline
                    Bandwidth(Mbps) & 10.9 & 11.4 & 10.7 & 10.9 & 10.8 & 10.8 & 10.7 & 10.5 & 10.6 & 10.6 & 10.6 & 10.5 & 11.1 & 12.1 & 11.7 \\
                    \hline   
                    \hline
                    Trial number & 16 & 17 & 18 & 19 & 20 & 21 & 22 & 23 & 24 & 25 & 26 & 27 & 28 & 29 & 30 \\ \hline 
                    Bandwidth(Mbps) & 11.7 & 11.5 & 11.7 & 11.5 & 10.7 & 10.4 & 10.6 & 10.5 & 11 & 10.8 & 12.1 & 10.4 & 12 & 10.9 & 10.6 \\ \hline    
                    \hline
                    Trial number & 31 & 32 & 33 & 34 & 35 & 36 & 37 & 38 & 39 & 40 & & & & & \\ \hline
                    Bandwidth(Mbps) & 11 & 11.7 & 12.2 & 11.4 & 12 & 11.7 & 12.3 & 11.9 & 12.1 & 12.2 & & & & &\\
                    \hline
                    \end{tabular}
                \end{center}
                \vspace{3mm}
            \paragraph{Results}
                \begin{itemize}
                    \item $\bar{x} = 11.2 Mbps$
                    \item $s = 0.62$
                    \item $Confidence$ $interval =(11.027 , 11.41) $
                    \item $N_{normal} = 181$
                \end{itemize}
            \vspace{3mm}           
             \begin{figure}[htp]
                \begin{center}
                    \subfigure[Qunatiltle quantile plot]{
                    \includegraphics[scale=0.5]{tcp-wifi.png}
                    }
                    \subfigure[Histogram plot]{
                    \includegraphics[scale=0.5]{tcp-wifi_hist.png}
                    }                        
                \end{center}    
            \end{figure}   
            \vspace{3mm}
            \paragraph{Observations}
                \begin{itemize}
                        \item Bandwidth is slower than that on lan network 
                        \item According quantile quantile plot the sample data roughly follow noraml distribution
                        \item There exist some noise in the data and this may be due to external factor as the transfer medium
                        \item The results makes more sense when using a wireless ad hoc network rather than that of traditional wifi network
                \end{itemize}
        \newpage        
        \subsubsection{ADSL Network}
            \paragraph{Readings}
                \begin{center}
                    \begin{tabular}{ ||l || c | c | c | c | c | c | c | c | c | c | c | c | c | c | c | }
                    \hline
                    Trial number & 1 & 2 & 3 & 4 & 5 & 6 & 7 & 8 & 9 & 10 & 11 & 12 & 13 & 14 & 15 \\ \hline
                    Bandwidth(Kbps) & 126 & 96.4 & 109 & 114 & 108 & 97.7 & 113 & 113 & 112 & 117 & 112 & 113 & 119 & 85 & 114  \\
                    \hline   
                    \hline
                    Trial number & 16 & 17 & 18 & 19 & 20 & 21 & 22 & 23 & 24 & 25 & 26 & 27 & 28 & 29 & 30 \\ \hline 
                    Bandwidth(Kbps) & 99.7 & 96.2 & 107 & 113 & 117 & 111 & 108 & 103 & 127 & 109 & 119 & 111 & 115 & 103 & 111  \\ \hline    
                    \hline
                    Trial number & 31 & 32 & 33 & 34 & 35 & 36 & 37 & 38 & 39 & 40 & & & & & \\ \hline
                    Bandwidth(Kbps) & 107 & 111 & 106 & 102 & 115 & 108 & 117 & 93.4 & 114 & 103 & & & & &\\
                    \hline
                    \end{tabular}
                \end{center}
                \vspace{3mm}
            \paragraph{Results}
                \begin{itemize}
                    \item $\bar{x} = 109.3 Kpbs$
                    \item $s = 8.43$
                    \item $Confidence$ $interval =(106.7 , 112) $
                    \item $N_{normal} = 229$
                \end{itemize}
            \vspace{3mm}           
             \begin{figure}[htp]
                \begin{center}
                    \subfigure[Qunatiltle quantile plot]{
                    \includegraphics[scale=0.5]{tcp-dsl.png}
                    }
                    \subfigure[Histogram plot]{
                    \includegraphics[scale=0.5]{tcp-dsl_hist.png}
                    }                        
                \end{center}    
            \end{figure}   
            \vspace{3mm}
            \paragraph{Observations}
                \begin{itemize}
                        \item Bandwidth is slower than both of lan and wlan nerworks
                        \item According quantile quantile plot the sample data roughly follow noraml distribution
                        \item Readings depends on the router of the server machine and its max bandwidth and whether its overloaded or not
                \end{itemize}
        \newpage
        \subsubsection{Dial up Network}
            \paragraph{Readings}
                \begin{center}
                    \begin{tabular}{ ||l || c | c | c | c | c | c | c | c | c | c | c | c | c | c | c | }
                    \hline
                    Trial number & 1 & 2 & 3 & 4 & 5 & 6 & 7 & 8 & 9 & 10 & 11 & 12 & 13 & 14 & 15 \\ \hline
                    Bandwidth(Kbps) & 87.9 & 87.0 & 99.5 & 106.0 & 93.2 & 64.8 & 101.0 & 106.0 & 101.0 & 100.0 & 106.0 & 94.2 & 93.4 & 100.0 & 114.0  
                    \\ \hline   
                    \hline
                    Trial number & 16 & 17 & 18 & 19 & 20 & 21 & 22 & 23 & 24 & 25 & 26 & 27 & 28 & 29 & 30 \\ \hline 
                    Bandwidth(Kbps) &  89.9 & 69.8 & 100.0 & 105.0 & 104.0 & 105.0 & 109.0 & 89.5 & 94.8 & 102.0 & 105.0 & 64.7 & 6.48 & 9.19 & 98.4  
                    \\ \hline    
                    \hline
                    Trial number & 31 & 32 & 33 & 34 & 35 & 36 & 37 & 38 & 39 & 40 & & & & & \\ \hline
                    Bandwidth(Kbps) & 112.0 & 22.0 & 57.4 & 66.9 & 71.6 & 108.0 & 47.4 & 59.4 & 41.8 & 32.9 & & & & &\\
                    \hline
                    \end{tabular}
                \end{center}
                \vspace{3mm}
            \paragraph{Results}
                \begin{itemize}
                    \item $\bar{x} = 83.1 Kpbs$
                    \item $s = 28.9$
                    \item $Confidence$ $interval =(74.2 , 92.1) $
                    \item $N_{normal} = 4632$
                \end{itemize}
            \vspace{3mm}           
             \begin{figure}[htp]
                \begin{center}
                    \subfigure[Qunatiltle quantile plot]{
                    \includegraphics[scale=0.5]{tcp-dailup.png}
                    }
                    \subfigure[Histogram plot]{
                    \includegraphics[scale=0.5]{tcp-dailup_hist.png}
                    }                        
                \end{center}    
            \end{figure}   
            \vspace{3mm}
            \paragraph{Observations}
                \begin{itemize}
                        \item Bandwidth is slowest among other networks
                        \item According quantile quantile plot we can assume that the sample data doesn't perfectly follow normal distribution
                 \end{itemize}

\newpage
% UDP SUBSECTION
    \subsection{UDP Connection}
    This batch script is used to collect the required data for UDP analysis using iperf
        \lstinputlisting{run.bat}
    	\vspace{3mm}

        \subsubsection{LAN Network}
            \paragraph{Readings}
                The following table includes the data collected for carrying out the analysis phase for the UDP connection on local area network (LAN). The data rates have been changed from 1 MBytes/sec up to 250 MBytes/sec with a step of 25 MBytes/sec in each iteration. For each data rate, three samples have been collected and the average throughput and loss rate has been recorded.
                \begin{figure}[htp]
                    \begin{center}
                        \subfigure[Readings for UDP Connection in LAN Network]{
                        \includegraphics[scale=0.6]{lan-udp-data.png}
                    }             
                    \end{center}    
                \end{figure}            
                \vspace{3mm}
                
            \paragraph{Results}
                The figures below illustrate the realation between the data rate in MBytes/sec and the acheived throughput and loss rate over a local area network (LAN).
                \begin{figure}[htp]
                    \begin{center}
                        \subfigure[Data Rate (MByte/sec)  vs Throughput (Mbps)]{
                            \includegraphics[scale=0.5]{lan-udp-throughput.png}
                        }
                        \subfigure[Loss Rate Percentage]{
                            \includegraphics[scale=0.5]{lan-udp-error.png}
                        }                        
                \end{center}    
            \end{figure}
            \paragraph{Observations}
                \begin{itemize}
                        \item Knee Point:
                            \begin{itemize}
                                \item Throughput: 25 Mbps
                                \item Data Rate: 30 MBytes/sec
                            \end{itemize}
                        \item Maximum Throughput: 37 Mbps
                        \item Compared to TCP (76.3 Mbps), it was expected that UDP acheives heigher throughput than TCP. However, due to the default packet size (1470 Bytes / datagram). By increasing the data packet size, the throughput for UDP was improved. Note however that differing nature of UDP and TCP flows means that it their measurements should not be directly compared. Iperf sends UDP datagrams are a constant steady rate, whereas TPC tends to send packet trains. This means that TCP is likely to suffer from congestion effects at a lower data rate than UDP. (\url{http://kb.pert.geant.net/PERTKB/IperfTool})
                        \item The error rate remains within the expected bound (Less than 1 percent). For the error rate below the knee point (30 MBytes/sec), the error increases with the increase in data rate. After the knee point, the error rate behavior tend to be unpredicted but remains heigher than 0.4 percent.
                        \item By changing the direct cable used to connect the two computer devices in the LAN, the UDP and TCP throughput was changed depending on the wire used. Higher quality wires achevied better results reaching up to 90.5 Mbps for TCP and 65 Mbps for UDP.
                \end{itemize}
        \newpage
        \subsubsection{WLAN Network}
            \paragraph{Readings}
                The following table includes the data collected for carrying out the analysis phase for the UDP connection on Wireless local area network (WLAN). The data rates have been changed from 2 MBytes/sec up to 20 MBytes/sec with a step of 2 MBytes/sec in each iteration. For each data rate, three samples have been collected and the average throughput and loss rate has been recorded.
                \begin{figure}[htp]
                        \begin{center}
                            \subfigure[Readings for UDP Connection in WLAN Network]{
                            \includegraphics[scale=0.6]{wireless-udp-data.png}
                        }             
                        \end{center}    
                \end{figure}            

                \vspace{3mm}

            \paragraph{Results}
                The figures below illustrate the realation between the data rate in MBytes/sec and the acheived throughput and loss rate over a wireless local area network (WLAN).
                \begin{figure}[htp]
                    \begin{center}
                        \subfigure[Data Rate (MByte/sec)  vs Throughput (Mbps)]{
                            \includegraphics[scale=0.5]{wireless-udp-throughput.png}
                        }
                        \subfigure[Loss Rate Percentage]{
                            \includegraphics[scale=0.5]{wireless-udp-error.png}
                        }                        
                \end{center}
            \end{figure}
            
            \vspace{3mm}
            
            \paragraph{Observations}
                \begin{itemize}
                        \item Knee Point:
                            \begin{itemize}
                                \item Throughput: 14.5 Mbps
                            \end{itemize}
                        \item Maximum Throughput: 14.5 Mbps
                        \item Compared to TCP (11.2 Mbps), The UDP connection acheived higher throughput compared to TCP (14.5 Mbps)
                        \item The error rate remains within the expected bound (Less than 0.4 percent). It's also clear that the loss rate increases with the increase in the data rate.
                \end{itemize}
        \newpage
        \subsubsection{ADSL Network}
            \paragraph{Readings}
                The following table includes the data collected for carrying out the analysis phase for the UDP connection using ADSL connection. The data rates have been changed from 25 KBytes/sec up to 300 KBytes/sec with a step of 25 KBytes/sec in each iteration. For each data rate, three samples have been collected and the average throughput and loss rate has been recorded.
                \begin{figure}[htp]
                        \begin{center}
                            \subfigure[Readings for UDP Connection in DSL Network]{
                            \includegraphics[scale=0.6]{dsl-udp-data.png}
                        }             
                        \end{center}    
                \end{figure}            

            \vspace{3mm}
            
            \paragraph{Results}
                The figures below illustrate the realation between the data rate in MBytes/sec and the acheived throughput and loss rate over a wireless local area network (WLAN).
                \begin{figure}[htp]
                    \begin{center}
                        \subfigure[Data Rate (KByte/sec)  vs Throughput (Kbps)]{
                            \includegraphics[scale=0.5]{dsl-udp-throughput.png}
                        }
                        \subfigure[Loss Rate Percentage]{
                            \includegraphics[scale=0.5]{dsl-udp-error.png}
                        }                        
                \end{center}
            \end{figure}
            
            \vspace{3mm}
            
            \paragraph{Observations}
                \begin{itemize}
                        \item Knee Point:
                            \begin{itemize}
                                \item Throughput: 135 Kbps
                            \end{itemize}
                        \item Maximum Throughput: 135 Kbps
                        \item Compared to TCP (109.3 Kbps), The UDP connection acheived higher throughput compared to TCP (135 Kbps)
                        \item It's clear that the loss rate increases with the increase in the data rate. The error reaches 80 Percent for high data transmission rates (300 KBytes/sec)
                \end{itemize}
                
\newpage                
    \section{Conclusion}
        \subsection{TCP}    
            \begin{figure}[htp]
                    \begin{center}
                        \subfigure[Probability plot]{
                        \includegraphics[scale=01]{tcp-all.png}
                        }                        
                    \end{center}    
                \end{figure}   
            \vspace{3mm}
            \begin{center}
                \begin{itemize}
                    \item Performance of wired network is better than that of wireless network
                    \item \[ Bandwidth_{LAN} > Bandwidth_{WLAN} > Bandwidth_{ADSL} > Bandwidth_{Dail up}\]
                    \item The distribution of the Dail up sample data follows the same ditribution as that of ADSL sample data which is not expected but the reason may be that during taking the readings of ADSL experiment the network was congested which cause this low results
                \end{itemize}
            \end{center}

\newpage

        \subsection{UDP}    
            \begin{figure}[htp]
                    \begin{center}
                        \subfigure[Comparison between UDP and TCP]{
                        \includegraphics[width=5cm]{comparison.png}
                        }                        
                    \end{center}    
                \end{figure}   
            \begin{center}
                \begin{itemize}
                    \item Performance of UDP connection is better than TCP connection. Although the data collected for the LAN network showed that TCP is faster than UDP, this is not the expected case. This could be due to the packet size assigned for the datagrams and several other different factors.
                    \item \[ Bandwidth_{LAN} > Bandwidth_{WLAN} > Bandwidth_{ADSL} > Bandwidth_{Dail up}\]
                    \item The packet loss ratio increases when the data rate increases.
		  \item It was quite difficult to measure the performance of UDP network over dial-up connection. This is due to the high loss ratio and failure to receive the final server report for the received datagram and loss rate.
                \end{itemize}
            \end{center}
                
                
% IF YOU'D RATHER TYPE THE CODE, OR HAVE A SMALLER BLOCK OF CODE, USE THIS:
%\begin{lstlisting}
%if(something)
%	do this
%else
%	do this
%\end{lstlisting}

%% THIS IS FROM A DIFFERENT CLASS, BUT DEMONSTRATES MATH MODE WELL
%%%%%%%%%%%%%%%%%%%%%%%%%%%%%%
\end{document}
